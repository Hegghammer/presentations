% ----Preamble----
\documentclass{beamer}
\usepackage{listings}
\usetheme{Hannover}

% ----Title page----
\title{Digital tools for qualitative research}
\author[Hegghammer]{Thomas Hegghammer}
\institute{thomas.hegghammer@ffi.no}
\date{NUPI, 29 October 2020}
\begin{document}

% ----Front slide----
\frame{\titlepage}

% ----Section title----
\section{Introduction}
% -----------------

% ----Slide-------
\begin{frame}
\frametitle{It doesn't have to be this way}
    \begin{columns}
        \column{0.5\textwidth}
            \begin{figure}
                \centering
                \includegraphics[width=4.75cm]{IJ2.jpg}
            \end{figure}
        \column{0.5\textwidth}
            \begin{figure}
                \centering
                \includegraphics[width=5cm]{Scientist2.jpg}
            \end{figure}
    \end{columns}
\end{frame}
% -----------------

% -----------------
\begin{frame}
\frametitle{So much to gain}
    \begin{itemize}
        \item \textbf{Efficiency}, \textbf{impact}, and \textbf{transparency}
        \item The "unwieldy sources problem"
        \item Do what you already do, just faster
        \item If you have learnt a foreign language, you can learn to code
        \item A little goes a long way
    \end{itemize}
\end{frame}
% -----------------

%------------------
\section{Capture}
% -----------------

% -----------------
\begin{frame}
\frametitle{Capture}
    \begin{itemize}
        \item Grab tidbits with browser extensions (Zotero/Evernote)
        \item Record \& transcribe interviews (Otter/Descript apps)
        \item Photograph archive docs (Dropbox/ABBYY apps)
        \item Scrape texts, images, video from web (R)
    \end{itemize}
\end{frame}
% -----------------

%------------------
\section{Retrieve}
% -----------------

% -----------------
\begin{frame}
\frametitle{Retrieve}
    \begin{itemize}
        \item Keep files in cloud for availability, sharing, and security
        \item Store refs in Zotero/Endnote; cite with plugins/BibTeX
        \item Place unwieldy material in Evernote/Devonthink/Pocket
        \item Reduce stress with a task manager (Omnifocus/Todoist)
        \item NB: Keep source depot, bib software \& task manager apart
    \end{itemize}
\end{frame}
% -----------------

%------------------
\section{Process}
% -----------------

% -----------------
\begin{frame}
\frametitle{Process}
    \begin{itemize}
        \item OCR archival material (R/ABBYY/etc)
        \item Extract text from PDFs (R/Pdftotext.com/etc) 
        \item Text mine documents or transcripts (R)
    \end{itemize}
\end{frame}
% -----------------

%------------------
\section{Write}
% -----------------

% -----------------
\begin{frame}
\frametitle{Write}
    \begin{itemize}
        \item Manage large manuscripts (Scrivener/Storyist)
        \item Unclutter your writing space (Ulysses/Write!/plaintext)
        \item Play with ideas (Mindnode/Coggle)
        \item Avoid distractions (Focus/StayFocusd/Cold Turkey/etc)
    \end{itemize}
\end{frame}
% -----------------

%------------------
\section{Collaborate}
% -----------------

% -----------------
\begin{frame}
\frametitle{Collaborate}
    \begin{itemize}
        \item Manage projects (Asana/Trello/etc)
        \item Keep track of ms versions (Dropbox/Google Docs/Git)
        \item Create a dynamic discussion space (Slack/Teams)
    \end{itemize}
\end{frame}
% -----------------

%------------------
\section{Visualize}
% -----------------

% -----------------
\begin{frame}
\frametitle{Visualize: \newline Figures and tables}
    \begin{columns}
        \column{0.5\textwidth}
            \begin{figure}
                \centering
                \includegraphics[width=5cm]{graphs.png}
            \end{figure}
        \column{0.5\textwidth}
            \begin{figure}
                \centering
                \includegraphics[width=5cm]{table.png}
            \end{figure}
    \end{columns}
\end{frame}
% -----------------

%------------------
\begin{frame}
\frametitle{Visualize: \newline Timelines and diagrams}
    \begin{figure}
        \centering
        \includegraphics[width=10cm]{azzamtimeline.png}
    \end{figure}
\end{frame}
% -----------------

%------------------
\begin{frame}
\frametitle{Visualize: \newline Maps}
    \begin{figure}
        \centering
        \includegraphics[width=4cm]{locations_south.png}
    \end{figure}
\end{frame}
% -----------------

%------------------
\begin{frame}
\frametitle{Visualize: \newline Interactive graphs}
    \begin{figure}
        \centering
        \includegraphics[width=6cm]{shiny.png}
    \end{figure}
    \begin{center}
        \small{\url{https://hegghammer.shinyapps.io/NRM_map/}}
    \end{center}
\end{frame}
% -----------------

%------------------
\section{Disseminate}
% -----------------

% -----------------
\begin{frame}
\frametitle{Disseminate}
    \begin{itemize}
        \item Post your publications on personal website
        \item Submit your best preprints to OSF/SocArxiv 
        \item Tweet for more impact
    \end{itemize}
\end{frame}
% -----------------

%------------------
\section{Make replicable}
% -----------------

% -----------------
\begin{frame}
\frametitle{Make replicable: \newline Active citation and ATI}
    \begin{columns}
        \column{0.5\textwidth}
            \begin{figure}
                    \centering
                    \includegraphics[width=5cm]{azzam_ac1.png}
            \end{figure}
            \tiny{\url{https://hegghammer.com/active-citation-azzam-and-palestine/}} \newline
            \newline
            \tiny{See also \url{azzambook.net/active-citation}}
        \column{0.5\textwidth}
            \begin{figure}
                    \centering
                    \includegraphics[width=5cm]{ATI.PNG}
            \end{figure}
            \tiny{\url{https://qdr.syr.edu/ati}}   
    \end{columns}
\end{frame}
% -----------------

%------------------
\begin{frame}
\frametitle{Make replicable: \newline Online repositories}
    \begin{figure}
        \centering
        \includegraphics[width=9cm]{JDR.PNG}
    \end{figure}
    \begin{center}
        \small{\url{www.jdr.as}}
    \end{center}
\end{frame}
% -----------------

%------------------
\begin{frame}
\frametitle{Make replicable: \newline Replication materials}
    \begin{figure}
        \centering
        \includegraphics[width=10cm]{Github.PNG}
    \end{figure}
\end{frame}
% -----------------

%------------------
\begin{frame}
\frametitle{Everything in moderation}
    \begin{itemize}
        \item Pen and paper is best for learning
        \item Notes can be photographed
        \item PowerPoint has problems (linearity, low resolution)
    \end{itemize}
\end{frame}
% -----------------

%------------------
\begin{frame}
\frametitle{Stimulating reads}
    \begin{itemize}
        \item \href{https://programminghistorian.org/}{The Programming Historian (Blog)}
        \item \href{https://www.r-bloggers.com/2017/03/the-5-most-effective-ways-to-learn-r/}{"The 5 Most Effective Ways to Learn R"} 
        \item David Allen, \href{https://www.amazon.co.uk/Getting-Things-Done-Stress-free-Productivity/dp/0349408947}{\textit{Getting Things Done}}
        \item \href{https://www.princeton.edu/~amoravcs/library/ps.pdf}{Andrew Moravcsik, "Active Citation: A Precondition for Replicable Qualitative Research" (PDF)}
        \item \href{https://www.inf.ed.ac.uk/teaching/courses/pi/2016_2017/phil/tufte-powerpoint.pdf}{Edward Tufte, \textit{The Cognitive Style of PowerPoint} (PDF)}
        \item \href{https://kieranhealy.org/publications/plain-person-text/}{Kieran Healy, \textit{The Plain Person’s Guide to Plain Text Social Science} (PDF)}
    \end{itemize}
\end{frame}
% -----------------

%------------------
\begin{frame}[fragile]
\frametitle{Under the hood}
R code for timeline in slide 10: \newline
\newline
\tiny{\begin{lstlisting}[showstringspaces=false, language=Python]
library(timelineS)

events <- c("Born in Jenin", "Joins MB", "To Damascus to study", 
            "Fled Palestine", "To Saudi to teach", "Back to Amman", 
            "Joins Fedayin", "To Egypt for PhD", "Starts at Jordan University", 
            "Emigrates to Mecca", "Moves to Islamabad", 
            "Founds Services Bureau", "Moves to Peshawar", "Assassinated")

dates <- as.Date(c("1941-11-14", "1953-07-01", "1962-09-01", "1967-06-10", 
            "1967-09-01", "1968-6-30", "1969-2-1", "1971-9-1", "1973-6-30", 
            "1980-10-31", "1981-11-30", "1984-10-1", "1986-6-1", 
            "1989-11-24"))

azzam <- data.frame(events, dates)

timelineS(azzam, main = "Main events in Abdallah Azzam's life", 
    line.width = 10, labels = paste(azzam[[1]]), point.color = "red", 
    label.font = 3, label.cex = 1, 
    label.length = c(0.2, 0.2, 0.5, 0.4, 0.3, 0.2, 0.6, 0.5, 0.4, 
    0.2, 0.5, 0.5, 0.2, 0.2))
\end{lstlisting}}
TeX code for this presentation: \url{https://github.com/Hegghammer/presentations}
\end{frame}
% -----------------

\end{document}